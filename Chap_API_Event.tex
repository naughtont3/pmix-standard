%%%%%%%%%%%%%%%%%%%%%%%%%%%%%%%%%%%%%%%%%%%%%%%%%
% Chapter: Events
%%%%%%%%%%%%%%%%%%%%%%%%%%%%%%%%%%%%%%%%%%%%%%%%%
\chapter{Event Notification}
\label{chap:api_event}

This chapter defines the \ac{PMIx} event notification system.
These interfaces are designed to support the reporting of events to/from clients and servers, and between library layers within a single process.

%%%%%%%%%%%%%%%%%%%%%%%%%%%%%%%%%%%%%%%%%%%%%%%%%
%%%%%%%%%%%%%%%%%%%%%%%%%%%%%%%%%%%%%%%%%%%%%%%%%
\section{Notification and Management}
\label{chap:api_event:notify}

\ac{PMIx} event notification provides an asynchronous out-of-band mechanism for communicating events between application processes and/or elements of the \ac{SMS}. Its uses span a wide range including fault notification, coordination between multiple programming libraries within a single process, and workflow orchestration for non-synchronous programming models. Events can be divided into two distinct classes:

\begin{itemize}
\item \textit{Job-specific events} directly relate to a job executing within the session, such as a debugger attachment, process failure within a related job, or events generated by an application process. Events in this category are to be immediately delivered to the \ac{PMIx} server library for relay to the related local processes.

\item \textit{Environment events} indirectly relate to a job but do not specifically target the job itself. This category includes \ac{SMS}-generated events such as \ac{ECC} errors, temperature excursions, and other non-job conditions that might directly affect a session's resources, but would never include an event generated by an application process. Note that although these do potentially impact the session's jobs, they are not directly tied to those jobs. Thus, events in this category are to be delivered to the \ac{PMIx} server library only upon request.
\end{itemize}

Both \ac{SMS} elements and applications can register for events of either type.

\adviceimplstart
Race conditions can cause the registration to come after events of possible interest (e.g., a memory \ac{ECC} event that occurs after start of execution but prior to registration, or an application process generating an event prior to another process registering to receive it). \ac{SMS} vendors are \textit{requested} to cache environment events for some time to mitigate this situation, but are not \textit{required} to do so. However, \ac{PMIx} implementers are \textit{required} to cache all events received by the \ac{PMIx} server library and to deliver them to registering clients in the same order in which they were received
\adviceimplend

\adviceuserstart
Applications must be aware that they may not receive environment events that occur prior to registration, depending upon the capabilities of the host \ac{SMS}.
\adviceuserend

The generator of an event can specify the \textit{target range} for delivery of that event. Thus, the generator can choose to limit notification to processes on the local node, processes within the same job as the generator, processes within the same allocation, other threads within the same process, only the \ac{SMS} (i.e., not to any application processes), all application processes, or to a custom range based on specific process identifiers. Only processes within the given range that register for the provided event code will be notified. In addition, the generator can use attributes to direct that the event not be delivered to any default event handlers, or to any multi-code handler (as defined below).

Event notifications provide the process identifier of the source of the event plus the event code and any additional information provided by the generator. When an event notification is received by a process, the registered handlers are scanned for their event code(s), with matching handlers assembled into an \textit{event chain} for servicing. Note that users can also specify a \textit{source range} when registering an event (using the same range designators described above) to further limit when they are to be invoked. When assembled, PMIx event chains are ordered based on both the specificity of the event handler and user directives at time of handler registration. By default, handlers are grouped into three categories based on the number of event codes that can trigger the callback:
\begin{itemize}
%
\item \textit{single-code} handlers are serviced first as they are the most specific. These are handlers that are registered against one specific event code.
%
\item \textit{multi-code} handlers are serviced once all single-code handlers have completed. The handler will be included in the chain upon receipt of an event matching any of the provided codes.
%
\item \textit{default} handlers are serviced once all multi-code handlers have completed. These handlers are always included in the chain unless the generator specifically excludes them.
%
\end{itemize}

Users can specify the callback order of a handler within its category at the time of registration. Ordering can be specified by providing the relevant event handler names, if the user specified an event handler name when registering the corresponding event. Thus, users can specify that a given handler be executed before or after another handler should both handlers appear in an event chain (the ordering is ignored if the other handler isn't included). Note that ordering does not imply immediate relationships. For example, multiple handlers registered to be serviced after event handler \textit{A} will all be executed after \textit{A}, but are not guaranteed to be executed in any particular order amongst themselves.

In addition, one event handler can be declared as the \textit{first} handler to be executed in the chain. This handler will \textit{always} be called prior to any other handler, regardless of category, provided the incoming event matches both the specified range and event code. Only one handler can be so designated --- attempts to designate additional handlers as \textit{first} will return an error. Deregistration of the declared \textit{first} handler will re-open the position for subsequent assignment.

Similarly, one event handler can be declared as the \textit{last} handler to be executed in the chain. This handler will \textit{always} be called after all other handlers have executed, regardless of category, provided the incoming event matches both the specified range and event code. Note that this handler will not be called if the chain is terminated by an earlier handler. Only one handler can be designated as \textit{last} --- attempts to designate additional handlers as \textit{last} will return an error. Deregistration of the declared \textit{last} handler will re-open the position for subsequent assignment.

\adviceuserstart
Note that the \textit{last} handler is called \textit{after} all registered default handlers that match the specified range of the incoming event unless a handler prior to it terminates the chain. Thus, if the application intends to define a \textit{last} handler, it should ensure that no default handler aborts the process before it.
\adviceuserend

Upon completing its work and prior to returning, each handler \textit{must} call the event handler completion function provided when it was invoked (including a status code plus any information to be passed to later handlers) so that the chain can continue being progressed. \ac{PMIx} automatically aggregates the status and any results of each handler (as provided in the completion callback) with status from all prior handlers so that each step in the chain has full knowledge of what preceded it. An event handler can terminate all further progress along the chain by passing the \refconst{PMIX_EVENT_ACTION_COMPLETE} status to the completion callback function.

\subsection{Events versus status constants}
\label{api:event:evssc}

Return status constants (see Section \ref{api:struct:errors}) represent values that can be returned from or passed into \ac{PMIx}
\acp{API}. These are distinct from \ac{PMIx} \emph{events} in that they are
not values that can be registered against event handlers. In general, the two
types of constants are distinguished by inclusion of an "ERR" in the name of
error constants versus an "EVENT" in events, though there are exceptions (e.g,
the \refconst{PMIX_SUCCESS} constant).


%%%%%%%%%%%%%%%%%%%%%%%%%%%%%%%%%%%%%%%%%%%%%%%%%
\subsection{\code{PMIx_Register_event_handler}}
\declareapi{PMIx_Register_event_handler}

%%%%
\summary

Register an event handler.

%%%%
\format

\copySignature{PMIx_Register_event_handler}{2.0}{
pmix_status_t \\
PMIx_Register_event_handler(pmix_status_t codes[], size_t ncodes, \\
\hspace*{28\sigspace}pmix_info_t info[], size_t ninfo, \\
\hspace*{28\sigspace}pmix_notification_fn_t evhdlr, \\
\hspace*{28\sigspace}pmix_hdlr_reg_cbfunc_t cbfunc, \\
\hspace*{28\sigspace}void *cbdata);
}

\begin{arglist}
\argin{codes}{Array of status codes (array of \refstruct{pmix_status_t})}
\argin{ncodes}{Number of elements in the \refarg{codes} array (\code{size_t})}
\argin{info}{Array of info structures (array of handles)}
\argin{ninfo}{Number of elements in the \refarg{info} array (\code{size_t})}
\argin{evhdlr}{Event handler to be called \refapi{pmix_notification_fn_t} (function reference)}
\argin{cbfunc}{Callback function \refapi{pmix_hdlr_reg_cbfunc_t} (function reference)}
\argin{cbdata}{Data to be passed to the cbfunc callback function (memory reference)}
\end{arglist}


If \refarg{cbfunc} is \code{NULL}, the function call will be treated as a \emph{blocking} call. In this case, the returned status will be either (a) the event handler reference identifier if the value is greater than or equal to zero, or (b) a negative error code indicative of the reason for the failure.

If the \refarg{cbfunc} is non-\code{NULL}, the function call will be treated as a \emph{non-blocking} call and will return the following:

\returnsimplenb
The result of the registration operation shall be returned in the provided callback function along with the assigned event handler identifier.

\returnstart
\begin{itemize}
\item \refconst{PMIX_ERR_EVENT_REGISTRATION} indicating that the registration
has failed for an undetermined reason.
\end{itemize}
\returnend

The callback function must not be executed prior to returning from the \ac{API}, and no events corresponding to this registration may be delivered prior to the completion of the registration callback function (\refarg{cbfunc}).

\reqattrstart
The following attributes are required to be supported by all \ac{PMIx} libraries:

\pasteAttributeItem{PMIX_EVENT_HDLR_NAME}
\pasteAttributeItem{PMIX_EVENT_HDLR_FIRST}
\pasteAttributeItem{PMIX_EVENT_HDLR_LAST}
\pasteAttributeItem{PMIX_EVENT_HDLR_FIRST_IN_CATEGORY}
\pasteAttributeItem{PMIX_EVENT_HDLR_LAST_IN_CATEGORY}
\pasteAttributeItem{PMIX_EVENT_HDLR_BEFORE}
\pasteAttributeItem{PMIX_EVENT_HDLR_AFTER}
\pasteAttributeItem{PMIX_EVENT_HDLR_PREPEND}
\pasteAttributeItem{PMIX_EVENT_HDLR_APPEND}
\pasteAttributeItem{PMIX_EVENT_CUSTOM_RANGE}
\pasteAttributeItem{PMIX_RANGE}
\pasteAttributeItem{PMIX_EVENT_RETURN_OBJECT}

\divider

Host environments that implement support for \ac{PMIx} event notification are required to support the following attributes when registering handlers - these attributes are used to direct that the handler should be invoked only when the event affects the indicated process(es):

\pasteAttributeItem{PMIX_EVENT_AFFECTED_PROC}
\pasteAttributeItem{PMIX_EVENT_AFFECTED_PROCS}

\reqattrend


%%%%
\descr

Register an event handler to report events. Note that the codes being registered do \textit{not} need to be \ac{PMIx} error constants --- any integer value can be registered. This allows for registration of non-PMIx events such as those defined by a particular \ac{SMS} vendor or by an application itself.

\adviceuserstart
In order to avoid potential conflicts, users are advised to only define codes that lie outside the range of the \ac{PMIx} standard's error codes. Thus, \ac{SMS} vendors and application developers should constrain their definitions to positive values or negative values beyond the \refconst{PMIX_EXTERNAL_ERR_BASE} boundary.
\adviceuserend


\adviceuserstart
As previously stated, upon completing its work, and prior to returning, each handler \textit{must} call the event handler completion function provided when it was invoked (including a status code plus any information to be passed to later handlers) so that the chain can continue being progressed. An event handler can terminate all further progress along the chain by passing the \refconst{PMIX_EVENT_ACTION_COMPLETE} status to the completion callback function. Note that the parameters passed to the event handler (e.g., the \refarg{info} and \refarg{results} arrays) will cease to be valid once the completion function has been called - thus, any information in the incoming parameters that will be referenced following the call to the completion function must be copied.
\adviceuserend

%%%%%%%%%%%%%%%%%%%%%%%%%%%%%%%%%%%%%%%%%%%%%%%%%
\subsection{Event registration constants}
\label{api:struct:constants:event}

\begin{constantdesc}
%
\declareconstitemvalue{PMIX_ERR_EVENT_REGISTRATION}{-144}
Error in event registration.
%
\end{constantdesc}

%%%%%%%%%%%%%%%%%%%%%%%%%%%%%%%%%%%%%%%%%%%%%%%%%
\subsection{System events}
\label{api:struct:sys:event}

\begin{constantdesc}
%
\declareconstitemNEW{PMIX_EVENT_SYS_BASE}
Mark the beginning of a dedicated range of constants for system event reporting.
%
\declareconstitemvalueNEW{PMIX_EVENT_NODE_DOWN}{-231}
A node has gone down - the identifier of the affected node will be included in the notification.
%
\declareconstitemvalueNEW{PMIX_EVENT_NODE_OFFLINE}{-232}
A node has been marked as \emph{offline} - the identifier of the affected node will be included in the notification.
%
\declareconstitemNEW{PMIX_EVENT_SYS_OTHER}
Mark the end of a dedicated range of constants for system event reporting.
%
\end{constantdesc}

\littleheader{Detect system event constant}
\declaremacro{PMIX_SYSTEM_EVENT}

Test a given event constant to see if it falls within the dedicated range of constants for system event reporting.

\copySignature{PMIX_SYSTEM_EVENT}{2.2}{
PMIX_SYSTEM_EVENT(a)
}

\begin{arglist}
\argin{a}{Error constant to be checked (\refstruct{pmix_status_t})}
\end{arglist}

Returns \code{true} if the provided values falls within the dedicated range of events for system event reporting.

%%%%%%%%%%%%%%%%%%%%%%%%%%%%%%%%%%%%%%%%%%%%%%%%%
\subsection{Event handler registration and notification attributes}
\label{api:struct:attributes:event}

Attributes to support event registration and notification.

%
\declareAttribute{PMIX_EVENT_HDLR_NAME}{"pmix.evname"}{char*}{
String name identifying this handler.
}
%
\declareAttribute{PMIX_EVENT_HDLR_FIRST}{"pmix.evfirst"}{bool}{
Invoke this event handler before any other handlers.
}
%
\declareAttribute{PMIX_EVENT_HDLR_LAST}{"pmix.evlast"}{bool}{
Invoke this event handler after all other handlers have been called.
}
%
\declareAttribute{PMIX_EVENT_HDLR_FIRST_IN_CATEGORY}{"pmix.evfirstcat"}{bool}{
Invoke this event handler before any other handlers in this category.
}
%
\declareAttribute{PMIX_EVENT_HDLR_LAST_IN_CATEGORY}{"pmix.evlastcat"}{bool}{
Invoke this event handler after all other handlers in this category have been called.
}
%
\declareAttribute{PMIX_EVENT_HDLR_BEFORE}{"pmix.evbefore"}{char*}{
Put this event handler immediately before the one specified in the \code{(char*)} value.
}
%
\declareAttribute{PMIX_EVENT_HDLR_AFTER}{"pmix.evafter"}{char*}{
Put this event handler immediately after the one specified in the \code{(char*)} value.
}
%
\declareAttribute{PMIX_EVENT_HDLR_PREPEND}{"pmix.evprepend"}{bool}{
Prepend this handler to the precedence list within its category.
}
%
\declareAttribute{PMIX_EVENT_HDLR_APPEND}{"pmix.evappend"}{bool}{
Append this handler to the precedence list within its category.
}
%
\declareAttribute{PMIX_EVENT_CUSTOM_RANGE}{"pmix.evrange"}{pmix_data_array_t*}{
Array of \refstruct{pmix_proc_t} defining range of event notification.
}
%
\declareAttribute{PMIX_EVENT_AFFECTED_PROC}{"pmix.evproc"}{pmix_proc_t}{
The single process that was affected.
}
%
\declareAttribute{PMIX_EVENT_AFFECTED_PROCS}{"pmix.evaffected"}{pmix_data_array_t*}{
Array of \refstruct{pmix_proc_t} defining affected processes.
}
%
\declareAttribute{PMIX_EVENT_NON_DEFAULT}{"pmix.evnondef"}{bool}{
Event is not to be delivered to default event handlers.
}
%
\declareAttribute{PMIX_EVENT_RETURN_OBJECT}{"pmix.evobject"}{void *}{
Object to be returned whenever the registered callback function \code{cbfunc} is invoked.
The object will only be returned to the process that registered it.
}
%
\declareAttribute{PMIX_EVENT_DO_NOT_CACHE}{"pmix.evnocache"}{bool}{
Instruct the \ac{PMIx} server not to cache the event.
}
%
\declareAttribute{PMIX_EVENT_PROXY}{"pmix.evproxy"}{pmix_proc_t*}{
\ac{PMIx} server that sourced the event.
}
%
\declareAttribute{PMIX_EVENT_TEXT_MESSAGE}{"pmix.evtext"}{char*}{
Text message suitable for output by recipient - e.g., describing the cause of the event.
}
%
\declareAttributeNEW{PMIX_EVENT_TIMESTAMP}{"pmix.evtstamp"}{time_t}{
System time when the associated event occurred.
}

%%%%%%%%%%%%%%%%%%%%%%%%%%%%%%%%%%%%%%%%%%%%%%%%%
\subsubsection{Fault tolerance event attributes}
\label{api:struct:attributes:ft}

The following attributes may be used by the host environment when providing an event notification as qualifiers indicating the action it intends to take in response to the event:

%
\declareAttribute{PMIX_EVENT_TERMINATE_SESSION}{"pmix.evterm.sess"}{bool}{
The \ac{RM} intends to terminate this session.
}
%
\declareAttribute{PMIX_EVENT_TERMINATE_JOB}{"pmix.evterm.job"}{bool}{
The \ac{RM} intends to terminate this job.
}
%
\declareAttribute{PMIX_EVENT_TERMINATE_NODE}{"pmix.evterm.node"}{bool}{
The \ac{RM} intends to terminate all processes on this node.
}
%
\declareAttribute{PMIX_EVENT_TERMINATE_PROC}{"pmix.evterm.proc"}{bool}{
The \ac{RM} intends to terminate just this process.
}
%
\declareAttribute{PMIX_EVENT_ACTION_TIMEOUT}{"pmix.evtimeout"}{int}{
The time in seconds before the \ac{RM} will execute the indicated operation.
}

%%%%%%%%%%%%%%%%%%%%%%%%%%%%%%%%%%%%%%%%%%%%%%%%%
\subsubsection{Hybrid programming event attributes}
\label{api:struct:attributes:hybrid}

The following attributes may be used by programming models to coordinate their use of common resources within a process in conjunction with the \refconst{PMIX_OPENMP_PARALLEL_ENTERED} event:
%
\pasteAttributeItem{PMIX_MODEL_PHASE_NAME}
\pasteAttributeItem{PMIX_MODEL_PHASE_TYPE}

%%%%%%%%%%%%%%%%%%%%%%%%%%%%%%%%%%%%%%%%%%%%%%%%%
\subsection{Notification Function}
\declareapi{pmix_notification_fn_t}

%%%%
\summary

The \refapi{pmix_notification_fn_t} is called by \ac{PMIx} to deliver notification of an event.

\adviceuserstart
The \ac{PMIx} \textit{ad hoc} v1.0 Standard defined an error notification function with an identical name, but different signature than the v2.0 Standard described below. The \textit{ad hoc} v1.0 version was removed from the v2.0 Standard is not included in this document to avoid confusion.
\adviceuserend


\copySignature{pmix_notification_fn_t}{2.0}{
typedef void (*pmix_notification_fn_t) \\
\hspace*{4\sigspace}(size_t evhdlr_registration_id, \\
\hspace*{5\sigspace}pmix_status_t status, \\
\hspace*{5\sigspace}const pmix_proc_t *source, \\
\hspace*{5\sigspace}pmix_info_t info[], size_t ninfo, \\
\hspace*{5\sigspace}pmix_info_t results[], size_t nresults, \\
\hspace*{5\sigspace}pmix_event_notification_cbfunc_fn_t cbfunc, \\
\hspace*{5\sigspace}void *cbdata);
}

\begin{arglist}
\argin{evhdlr_registration_id}{Registration number of the handler being called (\code{size_t})}
\argin{status}{Status associated with the operation (\refstruct{pmix_status_t})}
\argin{source}{Identifier of the process that generated the event (\refstruct{pmix_proc_t})}. If the source is the \ac{SMS}, then the nspace will be empty and the rank will be PMIX_RANK_UNDEF
\argin{info}{Information describing the event (\refstruct{pmix_info_t})}. This argument will be NULL if no additional information was provided by the event generator.
\argin{ninfo}{Number of elements in the info array (\code{size_t})}
\argin{results}{Aggregated results from prior event handlers servicing this event (\refstruct{pmix_info_t})}. This argument will be \code{NULL} if this is the first handler servicing the event, or if no prior handlers provided results.
\argin{nresults}{Number of elements in the results array (\code{size_t})}
\argin{cbfunc}{\refapi{pmix_event_notification_cbfunc_fn_t} callback function to be executed upon completion of the handler's operation and prior to handler return (function reference)}.
\argin{cbdata}{Callback data to be passed to cbfunc (memory reference)}
\end{arglist}

%%%%
\descr

Note that different \acp{RM} may provide differing levels of support for event notification to application processes. Thus, the \refarg{info} array may be \code{NULL} or may contain detailed information of the event. It is the responsibility of the application to parse any provided info array for defined key-values if it so desires.

\adviceuserstart
Possible uses of the \refarg{info} array include:

\begin{itemize}
%
\item for the host \ac{RM} to alert the process as to planned actions, such as aborting the session, in response to the reported event
%
\item provide a timeout for alternative action to occur, such as for the application to request an alternate response to the event
%
\end{itemize}

For example, the \ac{RM} might alert the application to the failure of a node that resulted in termination of several processes, and indicate that the overall session will be aborted unless the application requests an alternative behavior in the next 5 seconds. The application then has time to respond with a checkpoint request, or a request to recover from the failure by obtaining replacement nodes and restarting from some earlier checkpoint.

Support for these options is left to the discretion of the host \ac{RM}. Info keys are included in the common definitions above but may be augmented by environment vendors.
\adviceuserend

\advicermstart
On the server side, the notification function is used to inform the \ac{PMIx} server library's host of a detected event in the \ac{PMIx} server library. Events generated by \ac{PMIx} clients are communicated to the \ac{PMIx} server library, but will be relayed to the host via the \refapi{pmix_server_notify_event_fn_t} function pointer, if provided.
\advicermend


%%%%%%%%%%%%%%%%%%%%%%%%%%%%%%%%%%%%%%%%%%%%%%%%%
\subsection{\code{PMIx_Deregister_event_handler}}
\declareapi{PMIx_Deregister_event_handler}

%%%%
\summary

Deregister an event handler.

%%%%
\format

\copySignature{PMIx_Deregister_event_handler}{2.0}{
pmix_status_t \\
PMIx_Deregister_event_handler(size_t evhdlr_ref, \\
\hspace*{30\sigspace}pmix_op_cbfunc_t cbfunc, \\
\hspace*{30\sigspace}void *cbdata);
}

\begin{arglist}
\argin{evhdlr_ref}{Event handler ID returned by registration (\code{size_t})}
\argin{cbfunc}{Callback function to be executed upon completion of operation \refapi{pmix_op_cbfunc_t} (function reference)}
\argin{cbdata}{Data to be passed to the cbfunc callback function (memory reference)}
\end{arglist}

If \refarg{cbfunc} is \code{NULL}, the function will be treated as a \emph{blocking} call and the result of the operation returned in the status code.

If \refarg{cbfunc} is non-\code{NULL}, the function will be treated as a \emph{non-blocking} call.

\returnsimplenb

\begin{itemize}
\item \refconst{PMIX_OPERATION_SUCCEEDED}, returned when the request was immediately processed successfully - the \refarg{cbfunc} will \textit{not} be called.
\end{itemize}

The returned status code of \refarg{cbfunc} will be one of the following:

\begin{itemize}
\item \refconst{PMIX_SUCCESS} The event handler was successfully deregistered.
\item \refconst{PMIX_ERR_BAD_PARAM} The provided \refarg{evhdlr_ref} was unrecognized.
\item \refconst{PMIX_ERR_NOT_SUPPORTED} The \ac{PMIx} implementation does not support event notification.
\end{itemize}

%%%%
\descr

Deregister an event handler. Note that no events corresponding to the referenced registration may be delivered following completion of the deregistration operation (either return from the \ac{API} with \refconst{PMIX_OPERATION_SUCCEEDED} or execution of the \refarg{cbfunc}).

%%%%%%%%%%%%%%%%%%%%%%%%%%%%%%%%%%%%%%%%%%%%%%%%%
\subsection{\code{PMIx_Notify_event}}
\declareapi{PMIx_Notify_event}

%%%%
\summary

Report an event for notification via any
registered event handler.

%%%%
\format

\copySignature{PMIx_Notify_event}{2.0}{
pmix_status_t \\
PMIx_Notify_event(pmix_status_t status, \\
\hspace*{18\sigspace}const pmix_proc_t *source, \\
\hspace*{18\sigspace}pmix_data_range_t range, \\
\hspace*{18\sigspace}pmix_info_t info[], size_t ninfo, \\
\hspace*{18\sigspace}pmix_op_cbfunc_t cbfunc, void *cbdata);
}

\begin{arglist}
\argin{status}{Status code of the event (\refstruct{pmix_status_t})}
\argin{source}{Pointer to a \refstruct{pmix_proc_t} identifying the original reporter of the event (handle)}
\argin{range}{Range across which this notification shall be delivered (\refstruct{pmix_data_range_t})}
\argin{info}{Array of \refstruct{pmix_info_t} structures containing any further info provided by the originator of the event (array of handles)}
\argin{ninfo}{Number of elements in the \refarg{info} array (\code{size_t})}
\argin{cbfunc}{Callback function to be executed upon completion of operation \refapi{pmix_op_cbfunc_t} (function reference)}
\argin{cbdata}{Data to be passed to the cbfunc callback function (memory reference)}
\end{arglist}

If \refarg{cbfunc} is \code{NULL}, the function will be treated as a \emph{blocking} call and the result of the operation returned in the status code.

If \refarg{cbfunc} is non-\code{NULL}, the function will be treated as a \emph{non-blocking} call.

\returnsimplenb
Note that a successful call does \textit{not} reflect the success or failure of delivering the event to any recipients.

\returnstart
\begin{itemize}
\item \refconst{PMIX_OPERATION_SUCCEEDED}, returned when the request was immmediately processed successfully - the \refarg{cbfunc} will \textit{not} be called.
\end{itemize}
\returnend

\reqattrstart
The following attributes are required to be supported by all \ac{PMIx} libraries:

\pasteAttributeItem{PMIX_EVENT_NON_DEFAULT}
\pasteAttributeItem{PMIX_EVENT_CUSTOM_RANGE}
\pasteAttributeItem{PMIX_EVENT_DO_NOT_CACHE}
\pasteAttributeItem{PMIX_EVENT_PROXY}
\pasteAttributeItem{PMIX_EVENT_TEXT_MESSAGE}

\divider

Host environments that implement support for \ac{PMIx} event notification are required to provide the following attributes for all events generated by the environment:

\pasteAttributeItem{PMIX_EVENT_AFFECTED_PROC}
\pasteAttributeItem{PMIX_EVENT_AFFECTED_PROCS}

\reqattrend

\optattrstart
Host environments that support \ac{PMIx} event notification may offer notifications for environmental events impacting the job and for \ac{SMS} events relating to the job. The following attributes may optionally be included to indicate the host environment's intended response to the event:

\pasteAttributeItem{PMIX_EVENT_TERMINATE_SESSION}
\pasteAttributeItem{PMIX_EVENT_TERMINATE_JOB}
\pasteAttributeItem{PMIX_EVENT_TERMINATE_NODE}
\pasteAttributeItem{PMIX_EVENT_TERMINATE_PROC}
\pasteAttributeItem{PMIX_EVENT_ACTION_TIMEOUT}

\optattrend

%%%%
\descr

Report an event for notification via any registered event handler. This function can be called by any \ac{PMIx} process, including application processes, \ac{PMIx} servers, and \ac{SMS} elements. The \ac{PMIx} server calls this \ac{API} to report events it detected itself so that the host \ac{SMS} daemon distribute and handle them, and to pass events given to it by its host down to any attached client processes for processing. Examples might include notification of the failure of another process, detection of an impending node failure due to rising temperatures, or an intent to preempt the application. Events may be locally generated or come from anywhere in the system.

Host \ac{SMS} daemons call the \ac{API} to pass events down to its embedded \ac{PMIx} server both for transmittal to local client processes and for the host's own internal processing where the host has registered its own event handlers. The \ac{PMIx} server library is not allowed to echo any event given to it by its host via this \ac{API} back to the host through the \refapi{pmix_server_notify_event_fn_t} server module function. The host is required to deliver the event to all \ac{PMIx} servers where the targeted processes either are currently running, or (if they haven't started yet) might be running at some point in the future as the events are required to be cached by the \ac{PMIx} server library.

Client application processes can call this function to notify the \ac{SMS} and/or other application processes of an event it encountered. Note that processes are not constrained to report status values defined in the official \ac{PMIx} standard --- any integer value can be used. Thus, applications are free to define their own internal events and use the notification system for their own internal purposes.

\adviceuserstart
The callback function will be called upon completion of the
\code{notify_event} function's actions. At that time, any messages required for executing the operation (e.g., to send the notification to the local \ac{PMIx} server) will
have been queued, but may not yet have been transmitted. The caller is required to maintain the input
data until the callback function has been executed --- the sole purpose of the callback function is to indicate when the input data is no longer required.
\adviceuserend

%%%%%%%%%%%%%%%%%%%%%%%%%%%%%%%%%%%%%%%%%%%%%%%%%
\subsection{Notification Handler Completion Callback Function}
\declareapi{pmix_event_notification_cbfunc_fn_t}

%%%%
\summary

The \refapi{pmix_event_notification_cbfunc_fn_t} is called by event handlers to indicate completion of their operations.

\copySignature{pmix_event_notification_cbfunc_fn_t}{2.0}{
typedef void (*pmix_event_notification_cbfunc_fn_t) \\
\hspace*{4\sigspace}(pmix_status_t status, \\
\hspace*{5\sigspace}pmix_info_t *results, size_t nresults, \\
\hspace*{5\sigspace}pmix_op_cbfunc_t cbfunc, void *thiscbdata, \\
\hspace*{5\sigspace}void *notification_cbdata);
}

\begin{arglist}
\argin{status}{Status returned by the event handler's operation (\refstruct{pmix_status_t})}
\argin{results}{Results from this event handler's operation on the event (\refstruct{pmix_info_t})}
\argin{nresults}{Number of elements in the results array (\code{size_t})}
\argin{cbfunc}{\refapi{pmix_op_cbfunc_t} function to be executed when \ac{PMIx} completes processing the callback (function reference)}
\argin{thiscbdata}{Callback data that was passed in to the handler (memory reference)}
\argin{cbdata}{Callback data to be returned when \ac{PMIx} executes cbfunc (memory reference)}
\end{arglist}

%%%%
\descr

Define a callback by which an event handler can notify the \ac{PMIx} library that it has completed its response to the notification. The handler is \textit{required} to execute this callback so the library can determine if additional handlers need to be called. The handler shall return \refconst{PMIX_EVENT_ACTION_COMPLETE} if no further action is required. The return status of each event handler and any returned \refstruct{pmix_info_t} structures will be added to the \refarg{results} array of \refstruct{pmix_info_t} passed to any subsequent event handlers to help guide their operation.

If non-\code{NULL}, the provided callback function will be called to allow the event handler to release the provided info array and execute any other required cleanup operations.

%%%%%%%%%%%%%%%%%%%%%%%%%%%%%%%%%%%%%%%%%%%%%%%%%
\subsubsection{Completion Callback Function Status Codes}

The following status code may be returned indicating various actions taken by other event handlers.

\begin{constantdesc}
%
\declareconstitemvalue{PMIX_EVENT_NO_ACTION_TAKEN}{-331}
Event handler: No action taken.
%
\declareconstitemvalue{PMIX_EVENT_PARTIAL_ACTION_TAKEN}{-332}
Event handler: Partial action taken.
%
\declareconstitemvalue{PMIX_EVENT_ACTION_DEFERRED}{-333}
Event handler: Action deferred.
%
\declareconstitemvalue{PMIX_EVENT_ACTION_COMPLETE}{-334}
Event handler: Action complete.
%
\end{constantdesc}

%%%%%%%%%%%%%%%%%%%%%%%%%%%%%%%%%%%%%%%%%%%%%%%%%
